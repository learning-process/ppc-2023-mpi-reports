\section{Постановка задачи}
\textbf{Целью} данной работы является исследование алгоритма построения минимальной выпуклой оболочки для компонент бинарного изображения с использованием последовательного и параллельного подходов, а также сравнение этих двух подходов. Для достижения этой цели необходимо решить следующие \textbf{задачи}:
\begin{enumerate}
	\item реализовать последовательный алгоритм построения минимальной выпуклой оболочки для компонент бинарного изображения на языке C++;
	\item реализовать параллельный алгоритм построения минимальной выпуклой оболочки для компонент бинарного изображения на языке C++ с использованием технологии MPI;
	\item реализовать ряд тестов с использованием фреймворка Google Test для проверки корректности работы программ;
	\item провести вычислительные эксперименты для сравнения времени работы последовательного и параллельного подходов на разных входных данных и разном числе процессов;
	\item сделать выводы об эффективности и качестве каждого подхода.
\end{enumerate}