\section{Введение}
В данной лабораторной работе рассматривается задача построения выпуклой оболочки для компонент бинарного изображения. Эта задача относится к области компьютерного зрения, которая занимается анализом изображений и видео с помощью компьютерных алгоритмов. Компьютерное зрение имеет широкое применение в различных сферах, таких как медицина, робототехника, промышленность и безопасность.\n
Прежде чем перейти к самой лабораторной работе, стоит ввести термины, определяющие её суть.\n
\textbf{Бинарное изображение} — это изображение, состоящее из двух цветов, обычно черного и белого. Любое бинарное изображение состоит из компонент, количество которых может быть различно.\n
\textbf{Компонента бинарного изображения} — это связная область одного цвета, например, белая фигура на черном фоне. Чтобы мы могли работать с конкретной компонентой бинарного изображения, их (компоненты) необходимо отличать друг от друга. За это отвечает маркировка связных компонент бинарного изображения, т.е. процесс присвоения уникальных номеров каждой компоненте.\n
\textbf{Выпуклая оболочка множества точек на плоскости} — это наименьшее выпуклое множество, содержащее все эти точки. Построение выпуклой оболочки для компонент бинарного изображения позволяет выделить их форму и границы, а также получить полезную информацию о их размере, ориентации, количестве и т.д.