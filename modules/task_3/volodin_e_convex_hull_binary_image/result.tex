\section{Вывод}
По графикам, полученным в ходе проведения экспериментов, можно сделать вывод о том, что параллельный алгоритм работает быстрее, чем последовательный.\n
Да, можно заметить, что при небольшом количестве точек в изображении параллельный алгоритм, особенно если число процессов велико, проигрывает в производительности последовательному, но это происходит потому что большая часть времени уходит на накладные расходы (создание процессов, пересылка данных между ними).\n
Однако, при росте размера входного изображения и при увеличении числа процессов можно заметить ускорение параллельного алгоритма относительно последовательного. Это означает, что чем больше размер исходного набора точек и чем больше число процессов, тем более быстро будет работать параллельный алгоритм относительно последовательного.\n
К последнему предложению стоит сделать небольшое замечание, касающееся того, что при очень большом числе процессов параллельный алгоритм может потерять свое ускорение, поскольку, как было сказано ранее, увеличиваются накладные расходы. Поэтому к выбору числа процессов тоже нужно подходить осторожно.